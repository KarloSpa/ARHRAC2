

\documentclass{studosi-workbook}

\begin{document}
   
    \komentar[Tetak45]{Popunio samo točne odgovore}
	
	\begin{zadatak} [MI-2020]
		U posljednjih 20 godina napon napajanja procesora:
		
		\begin{zaokruzivanje}
			\ponuda*{smanjuje se i s 5V dostiže oko 1V}
			\ponuda{ništa od navedenog}
		\end{zaokruzivanje}
	\end{zadatak}


	\begin{zadatak} [MI-2020]
	    Programsko brojila uvećava se u pravilu:
		
		\begin{zaokruzivanje}
			\ponuda*{tijekom faze pribavi}
			\ponuda{ništa od navedenog}
		\end{zaokruzivanje}
	\end{zadatak}
	
	
	\begin{zadatak} [MI-2020]
		Kada 8-instrukcijski procesor sadržaj registra MDR prosljeđuje u akumulator?
		
		\begin{zaokruzivanje}
			\ponuda*{kod nardebe LDA}
			\ponuda{ništa od navedenog}
		\end{zaokruzivanje}
	\end{zadatak}
	
	
	\begin{zadatak} [MI-2020]
		Tipčno mikroprogram koji implementira fazu izvrši makroinstrukcije završava:
		
		\begin{zaokruzivanje}
			\ponuda*{pozivom mikroprograma za fazu pribavi}
			\ponuda{ništa od navedenog}
		\end{zaokruzivanje}
	\end{zadatak}
	
	
	\begin{zadatak} [MI-2020]
		Koji od navednih razloga ne otežava porast performansa novih procesora?
		
		\begin{zaokruzivanje}
			\ponuda*{Nedovoljan broj tranzistora na integriranom sklopu}
			\ponuda{ništa od navedenog}
		\end{zaokruzivanje}
	\end{zadatak}
	
	
	\begin{zadatak} [MI-2020]
        Za sklopove s 3 stanja vrijedi:		
		\begin{zaokruzivanje}
			\ponuda*{da se izlazi takvih sklopova mogu kratko spojiti}
			\ponuda{ništa od navedenog}
		\end{zaokruzivanje}
	\end{zadatak}
	
	
	\begin{zadatak} [MI-2020]
		Negativni brojevi u notaciji dvojnog komplementa
		
		\begin{zaokruzivanje}
			\ponuda*{imaju najznačajniji bit = 1}
			\ponuda{ništa od navedenog}
		\end{zaokruzivanje}
	\end{zadatak}
	
	
	\begin{zadatak} [MI-2020]
		Na kojoj 16-bitnoj adresi se nalazi memorijski operand instrukcije STA \$06, procesora M6800
		
		\begin{zaokruzivanje}
			\ponuda*{\$0006}
			\ponuda{ništa od navedenog}
		\end{zaokruzivanje}
	\end{zadatak}
	
	
	\begin{zadatak} [MI-2020]
		Ključni element sklopovske izvedbe stoga je:
		
		\begin{zaokruzivanje}
			\ponuda*{posmačni registar}
			\ponuda{ništa od navedenog}
		\end{zaokruzivanje}
	\end{zadatak}
	
	
	\begin{zadatak} [MI-2020]
		Mikroprogramirani procesor s predavanja u svakom µciklusu omogućava sljedeće aritmetičke operacije:
		
		\begin{zaokruzivanje}
			\ponuda*{i zbrajanje i posmak}
			\ponuda{ništa od navedenog}
		\end{zaokruzivanje}
	\end{zadatak}
	
	
	\begin{zadatak} [MI-2020]
		Za tipične horizontalne mikroinstrukcije vrijedi:
		
		\begin{zaokruzivanje}
			\ponuda*{}
			\ponuda{ništa od navedenog}
		\end{zaokruzivanje}
	\end{zadatak}
	
	
	\begin{zadatak} [MI-2020]
		Glavni nedostatak Von Neumannovog modela je:
		
		\begin{zaokruzivanje}
			\ponuda*{memorijsko usko grlo}
			\ponuda{ništa od navedenog}
		\end{zaokruzivanje}
	\end{zadatak}
	
	
	\begin{zadatak} [MI-2020]
		Koje podatke procesor MC68000 sprema prilikom obrade iznimke:
		
		\begin{zaokruzivanje}
			\ponuda*{samo programsko brojilo i registar stanja}
			\ponuda{ništa od navedenog}
		\end{zaokruzivanje}
	\end{zadatak}
	
	
	\begin{zadatak} [MI-2020]
		Koji slijed bajtova kodira instrukciju add \$daed pod pretpostavkama da se radi o 8-bitnom računalu, da je operacijski kod \$ce, te da se koristi little endian
		
		\begin{zaokruzivanje}
		    \ponuda{\$ daedce}
		    \ponuda{\$ eddace}
			\ponuda{\$ daceed}
			\ponuda*{\$ ceedda}
		\end{zaokruzivanje}
	\end{zadatak}
	
	
	\begin{zadatak} [MI-2020]
		Emit polje u mikroinstrukciji predstavlja:
		
		\begin{zaokruzivanje}
			\ponuda*{područje za definiranje konstante u mikroprogramu}
			\ponuda{ništa od navedenog}
		\end{zaokruzivanje}
	\end{zadatak}
	
	
	\begin{zadatak} [MI-2020]
		Memorijski prostor za parametre u potprogramu u jeziku C tipično:
		
		\begin{zaokruzivanje}
			\ponuda*{zauzima pozivatelj i otpušta pozivatelj}
			\ponuda{ništa od navedenog}
		\end{zaokruzivanje}
	\end{zadatak}
	
	
	\begin{zadatak} [MI-2020]
		2 RAM modula nakon ROM-a (4KiB), RAM ima priključke A0-A11, D0-D7, E, E*, R/W*.
		Spajanje na 16bitnu adresnu i osmobitnu podatkovnu sabirnicu?
	\end{zadatak}	
	
	
	\begin{zadatak} [MI-2020]
		Razmatramo petlju kroz koju program prolazi 100 ilijuna puta. Tijelo petlje ima 40 strojnih instrukcija. Petlju izvodimo na računalima A i B koja imaju radni takt 2GHz. Jedina razlika između 2 računala je u tome što računalo B cjelobrojne instrukcije izvodi dvaput brže od računala A. Na računalu A izvođenje petlje traje 1s, a na računalu B 0.8s.
		
		\begin{zaokruzivanje}
			\ponuda{Koliki je prosječni CPI na računalu A?}
			\ponuda{Koliki je udio cjelobrojnih instrukcija u tijelu petlje?}
		\end{zaokruzivanje}
	\end{zadatak}	
	
	
	\begin{zadatak} [MI-2020]
		U 8-instrukcijskom procesoru implementirati makronaredbu shl X
	\end{zadatak}	
	
	
	\begin{zadatak} [MI-2020]
		Program počinje od adrese \$0100
		\newline
		LOOP:  \char0009 \char0009 \char0009  DECA; op. kod 4A
		\newline
	    \char0009 \char0009 \char0009 \char0009 \char0009 
	    \char0009 \char0009 \char0009 \char0009 \char0009 BNE LOOP; op. kod 26
	    \newline
	    
	    DECA - dekrementira akumulator
	    \newline
	    BNE  - branch if not 0
	    \newline
	    A = 02
	    \newline
	    Skicirati memoriju i stanja na sabirnicama
	    \newline	
	\end{zadatak}
	
	
	\begin{zadatak} [MI-2020]
		Model mikroprogramiranog procesora sa labosa ak. godine 2020./2021.\newline
		Registre r4 i r5 nije potrebno sačuvati između dvije instrukcije, a r7 je programsko brojilo.\newline
		Napišite mikrokod POW ri, rj koja u ri pohranjuje \[2^{rj}\]
		U slučaju preljeva, ri treba postaviti na -1.\newline
		
		Dana je skica procesora i Memory Interface odsječak uputa
	\end{zadatak}
	
\end{document}