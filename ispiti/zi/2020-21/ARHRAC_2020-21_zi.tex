

\documentclass{studosi-workbook}

\begin{document}
   
    \komentar[Tetak45]{Popunio samo točne odgovore}
	
	\begin{zadatak} [ZI-2021]
		Zašto se tipično pojavljuju hazardi WAR i WAW
		
		\begin{zaokruzivanje}
			\ponuda*{zbog ograničenog broja registara programskog modela}
			\ponuda{zbog uvjetnog grananja}
			\ponuda{zbog varijabilnog formata instrukcija}
			\ponuda{zbog složenih načina adresiranja}
		\end{zaokruzivanje}
	\end{zadatak}
     
    \begin{zadatak} [ZI-2021]
		Kada se događa iznimka pogreške stranice
		
		\begin{zaokruzivanje}
			\ponuda{idk man}
		\end{zaokruzivanje}
	\end{zadatak}
	
	\begin{zadatak} [ZI-2021]
		Koja od slijedećih tehnika može dovesti do bolje popunjenosti protočne strukture superskalarne arhitekture sa statičkim izdavanjem
		
		\begin{zaokruzivanje}
			\ponuda{smanjenje broja funkcijskih jedinica}
			\ponuda*{razvijanje petlje (možda točno?)}
			\ponuda{izvođenje izvan redoslijeda}
			\ponuda{neprotočna izvedba}
		\end{zaokruzivanje}
	\end{zadatak}
    
	\begin{zadatak} [ZI-2021]
		Tipična protočna troadresna RISC arhitektura u svakom ciklusu sigala takta obavlja
		
		\begin{zaokruzivanje}
			\ponuda{točno 2 pristupa registarskom skupu}
			\ponuda{točno 3 pristupa registarskom skupu}
			\ponuda{najviše 2 pristupa registarskom skupu}
			\ponuda*{najviše 3 pristupa registarskom skupu}
			
		\end{zaokruzivanje}
	\end{zadatak}
	
	\begin{zadatak} [ZI-2021]
    	Kod dinamičkog raspoređivanja redoslijed izdavanja ovisi o ...
		
		\begin{zaokruzivanje}
			\ponuda{strukturnim hazardima}
			\ponuda{isključivo o vrstama instrukcija}
			\ponuda{isključivo o adresnom načinu instrukcija}
			\ponuda*{o tome nalazi li se CPU u nadgledanom načinu rada (možda točno?)}
			
		\end{zaokruzivanje}
	\end{zadatak}
	
	\begin{zadatak} [ZI-2021]
    	U kojem od navedenih stanja petlja ne može biti?
		
		\begin{zaokruzivanje}
			\ponuda*{spekulativno}
			\ponuda{aktivno}
			\ponuda{pripravno}
			\ponuda{blokirano}
			
		\end{zaokruzivanje}
	\end{zadatak}
	
	\begin{zadatak} [ZI-2021]
    	Zrnatost zaštite pristupa kod straničenja je 
		
		\begin{zaokruzivanje}
			\ponuda{na razini procesa}
			\ponuda{na razini memorijske lokacije}
			\ponuda{na razini segmenta}
			\ponuda*{na razini stranice (možda točno?)}
			
		\end{zaokruzivanje}
	\end{zadatak}	
	
	\begin{zadatak} [ZI-2021]
    	Koja od navednih veličina nije parametar priručne memorije
		
		\begin{zaokruzivanje}
			\ponuda{veličina linije}
			\ponuda{algoritam zamjene}
			\ponuda*{veličina stranice (možda točno)}
			\ponuda{asocijativnost}
			
		\end{zaokruzivanje}
	\end{zadatak}
	
	\begin{zadatak} [ZI-2021]
    	Koja ne postoji 
		
		\begin{zaokruzivanje}
			\ponuda{UMA}
			\ponuda{COMA}
			\ponuda*{NOMA (bez priručne memorije)}
			\ponuda{NUMA}
			
		\end{zaokruzivanje}
	\end{zadatak}
	
	\begin{zadatak} [ZI-2021]
    	Koju od slijedećih operacija s usputnom konstantom put podataka temeljne arhitekture MIPS *ne podržava*
		
		\begin{zaokruzivanje}
			\ponuda{relativno grananje}
			\ponuda{zbrajanje s konstantom}
			\ponuda*{množenje s konstantom (možda točno?)}
			\ponuda{zbrajanje s konstantom}
			
		\end{zaokruzivanje}
	\end{zadatak}
	
	\begin{zadatak} [ZI-2021]
    	Tijekom ciklusa puta podataka temeljne arhitekture MIPS, podatak pročitan iz podatkovne memorije može se proslijediti
		
		\begin{zaokruzivanje}
			\ponuda*{ulaznoj sabirnici registarskog skupa (možda točno?)}
			\ponuda{izlaznoj sabirnici registarskog skupa}
			\ponuda{sklopu za upravljanje grananjem}
			\ponuda{zbrajalu}
			
		\end{zaokruzivanje}
	\end{zadatak}
	
	\begin{zadatak} [ZI-2021]
    	Zašto jednostavna procjena prosječnog pristupa memoriji nije relevantna za moderne procesore opće namjene
		
		\begin{zaokruzivanje}
			\ponuda*{zbog dinamičkog raspoređivanja (možda točno?)}
			\ponuda{zbog predviđanja grananja}
			\ponuda{zbog priručnih memorija}
			\ponuda{zbog preimenovanja registara}
			
		\end{zaokruzivanje}
	\end{zadatak}
	
\end{document}